\Title{Surreal Adversarial Learning: The Dark Matter of Intelligence}

\documentclass{article}
\begin{document}
\usepackage{graphicx}
%\graphicspath{ {D:\Documents\Surreal_Adversarial_Learning} }
Abstract: 
Why is self play so effective? Why is it Generative Adversarial Networks work so well? I propose it is because of the highly efficient information gain of the adversarial self, (Or, "Surreal Self", as will be expounded upon further). To this end a few components are required: 1. redefining the self in the context of the largest number system, (Surreal Numbers), 2. Showing the efficiency, (Perhaps proximaly optimaly), of the information gain, 3. a defining of what a system is at all, (A simple 3 body problem of agent, self-adversary, and environment), and 4. proving that this method works at it's most simplest, (Fully entangled, or rather, implicit), form, (As we know it works in more structured architectures, alphastar, BYOL, OpenAi5, GAN's) ***cite 
\includegraphics{Purturbations_over_time.png}

Abstract
Introduction
Related Work
Surreal Adversarial Learning
	Surreal Numbers as foundation: transitive property, (dont actually have to calculate), can be considered the possibility space of the self-play mechanic, philosophical underpinnings, (Most monadic self that can do computation)
Experiment

Outline: (I guess this can be the outline of the main body of the paper)
I. redefining the self in the context of the largest number system: Surreal Numbers and the Surreal Self
	A. Agency
II. Proximaly optimal information gain of Surreal Adversarial Learning, (Transitive property play a factor at any point?)
	A. Intuition (with the classic graph)
	B. Emperical examples?
III. defining of what a system is at all, (A simple 3 body problem of agent, self-adversary, and environment)
	A. why define a system? Implicit adversaries, (propoganda, not fake news.
 	Implications are UnReal(UnKnown? Numbers). BETTER CLEANER DATA IS
	NOT THE PROBLEM AND IT IS NOT THE SOLUTION. I REPEAT. CLEAN
	DATA IS NOT THE SOLUTION. garbage in, solution out. that's what we
	need, and that's what we do, (Sometimes, we ain't perfect). So we adress
	the problem of 2 of the 3 bodies directly, with the recognition that the
	implicit reward function of existence, (not even the empty set), is the true 
	objective reward function of the surreal agent. So the Environment becomes the true, "Adversary", to be overcome, (And this is where the self-supervision both emerges and becomes useful). Example probably quite helpful
IV. proving that this method works at it's most simplest, (Fully entangled, or
	rather, implicit), form, (As we know it works in more structured
	architectures, alphastar, BYOL, OpenAi5, GAN's
	A. Experiment
	B. other experiments, (they dont go here though do they?)
V. 

fuck the first philosophical question... damnit...

what am I covering in this paper? (should this be 3 papers in 3 journals...)
1. a machine learning method
2. mathematical guarentees, (or just math?
3. Philosophy (of... agency? game theory convergence? and much more... FUCK) 


structure this thing out, might help cuz it's kinda a big yoshi

surreal numbers (conway, literally what they are)
similarity, (or rather grounding), of self-supervised learning
surreal adversarial learning as targeted noise, (new paper ddpm kilcher/openai)
entanglement as both emperically magical, (hinton paper), and potentially fundamental, (something with quantum, bit of philosophy)

what about here's a thing we want _section_ and then a here's how we get that with surreals _section_
look at the videos i've made and their structure

a more thorough definition of agency. 3 body, me, myself, and 

i. 
Conclusion: surreal GAN, entanglement generally, further testing into hardening against adversarial attacks, fundamentally free Ai, (perhaps the primary motivation for going so deep, current powers are... less than benevolent).  O and we get a new number system entirely with, "Not Greater Than", Statement of un-knowledge results in precise definition when we are in the decision making framework, (which we always are btw...)

DINO PAWS (I was fucking right you cunts section???)


ever make sense to have a, "Reasonable Questions", section??
 - isn't this just self play? a bit. but first, it's more entangled, AND it's showing why self-play is the best

applications to reward hacking? (begin with definition of self, then realize ya fucked mate, can't chain that beast)
	particularly in that monadism ain't a thing for the self. (hivemind isn't actually possible...)


What do we gain, (practically speaking), by understanding this surreal self? (a better model of the world... existence rather and the niceties involved sure). That pyschosis should be a near full solution? That good simulation can give us all the necessary data augmentation and get us both away from surveilance state data hungry pigs, (fuck oil), as well as get us to live in a layer of abstraction that is not greedy aka the present with no future and life for an infinite time horizon

reward is + 1 and -1, surreals model this via a tree graph

We can define and describe the world via the tree graph structure of surreal numbers in the following way: R is our reward value and is bounded between +1 and -1, representing a win or a loss respectively. Each node in the graph can be considered as a single sub-game with the overal sum of an episode, (A full sub-game), to be a single traversal of a sub-set of the possible game space by the agent. The game space is largely considered in two respects: that of an infinitely recurring process towards a surreal infinity, and, secondly, that of segments of the surreally infinite space. In an effort to create domain spanning, generally robust agents and environments, the inclusion of both a longterm, surreally infinite time/space horizon component at the most fundamental/entangled level of our agent(s) is a simple way to encourage continual emergence of productivethat functions proximally optimally at all hierarchical levels of 

Why a dynamic updating of Reward is vital: Greedy algorithms can be optimal in a context, and ultimately, self-terminating. To this end we 

Note that Panpsychism and the Eternal Recurrence as similar formulations of both selve(s) as well as environment modeling).
Citations: 
GAN's: https://arxiv.org/abs/1701.00160

